Most character animation in interactive applications is based on Skeletal-Subspace Deformation methods (see \cite{Magnenat-thalmann}). Several methods have been proposed providing finer results for hand meshes. Physically based muscle models have been used to model movement of skin and bones \cite{Fernando:2004:GGP:983868}. Data-driven methods have also been applied successfully, such as using eigenvectors to enhance SSD capabilities at very little cost \cite{Kry:2002:ERT:545261.545286}, or a set of control points to decouple the overall shape and wrinkle reconstructions \cite{Huang:2012:DCD:2311637.2311785}. Yet, how to realistically light these hand meshes is left untouched.

Skin is constituted by layers, and none is perfectly opaque. As such, subsurface scattering should be taken into account. Tattoos, freckles and veins are specially affected by this effect. While there are hardware based techniques for real time rendering of subsurface scattering \cite{Fernando:2004:GGP:983868}, they rely on approximations which are mostly intended for coarse effects, making them ineffective for these fine details.

Variations in the blood flow are also responsible for strong variations in skin color. Significant work has occurred for the rendering of this variation. Simulation of melanin and hemoglobin levels has been successfully used to represent changes of color due to facial expressions, emotions \cite{Jimenez:2010:PAM:1882261.1866167} or pressure \cite{Donner:2008:LHR:1409060.1409093}. Yet, this requires the capture of melanin and hemoglobin distribution from \emph{in vivo} scenarios, and has limited usability on interactive environments.


Additional note: \cite{Heidrich:2000:IMG:344779.344984} as information on how to reconstruct geometry from a normal map and precompute it's visibility. Still not sure what to make of it.