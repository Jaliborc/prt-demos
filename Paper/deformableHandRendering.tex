%%% The parameter given to the ``acmsiggraph'' LaTeX class in the
%%% ``\documentclass'' command controls several features of the formatted
%%% output: the presence or absence of hyperlinked icons just prior to the
%%% first section of the paper, the amount of space left clear for the ACM
%%% copyright notice, the presence or absence of line numbers and submission
%%% ID, and the presence or absence of an appropriate ``preprint'' notice.
%%%
%%% If you are preparing a paper for presentation in the Technical Papers
%%% program at one of our two annual flagship conferences, held in North
%%% America (SIGGRAPH) or Asia (SIGGRAPH Asia), you should use ``annual''
%%% as the parameter.
%%%
%%% If you are preparing a paper for presentation at one of our sponsored
%%% events, including SIGGRAPH and SIGGRAPH Asia, but not in those events'
%%% Technical Papers program, you should use ``sponsored'' as the parameter.
%%% (Technical Briefs and Game Papers presented at our annual flagship
%%% events fall into this category, as do papers accepted to other SIGGRAPH-
%%% sponsored events, such as I3D or ETRA or VRCAI.)
%%%
%%% If you are preparing a version of your content for review, you should
%%% use ``review'' as the parameter. Line numbers will be added to your
%%% paper, and the submission ID value will be printed across the top of
%%% each page of your paper. (Use the submission ID as the parameter to the
%%% ``TOGonlineID'' command, below.)
%%%
%%% If you are preparing an abstract, typically one to four pages in
%%% length, you should use ``abstract'' as the parameter. No space will
%%% be left clear for the ACM copyright notice, as copyright is not
%%% transferred for abstracts. A small permission notice will be added
%%% to your content during production in the footer of the first page.
%%%
%%% If you are preparing a preprint of your content, you should use
%%% ``preprint'' as the parameter. This is primarily for annual conference
%%% papers; a header reading ``To appear in ACM TOG X(Y)'' will appear on
%%% each page of the formatted output (where X is the volume and Y is the
%%% number of the issue in which it will be published).

%\documentclass[annual]{acmsiggraph}
\documentclass[review]{acmsiggraph}

\usepackage{comment}

%%% Definitions and commands that begin with ``\TOG'' are meant to be used
%%% in the preparation of papers to be presented in the Technical Papers
%%% program at one of our annual flagship events - SIGGRAPH and SIGGRAPH
%%% Asia. You can safely ignore these definitions and commands if your
%%% content is to be presented in some other venue.

%%% ``\TOGonlineid'' should be filled with the online ID value you received
%%% when you submitted your technical paper. It will be printed out if you
%%% prepare a ``review'' version of your paper.

\TOGonlineid{45678}

%%% Should your technical paper be accepted, you will be given three pieces
%%% of information: the volume and number of the issue of the ACM Transactions
%%% on Graphics journal in which your paper will be published, and the
%%% ``article DOI'' value, which is unique to your paper and provides the
%%% link to your paper's page in the ACM Digital Library. Fill in the
%%% ``\TOGvolume,'' ``\TOGnumber,'' and ``\TOGarticleDOI'' definitions with
%%% the three pieces of information you receive.

\TOGvolume{0}
\TOGnumber{0}
\TOGarticleDOI{1111111.2222222}

%%% By default, your technical paper will contain hyperlinked icons which
%%% point to your paper's article page in the ACM Digital Library, and to
%%% the paper itself in the ACM Digital Library. You may wish to add one
%%% or more links to your own resources. If any of the following four
%%% definitions have URLs in them, an appropriate hyperlinked icon will be
%%% added to the list.

\TOGprojectURL{}
\TOGvideoURL{}
\TOGdataURL{}
\TOGcodeURL{}

%%% Define the title of your paper here. Use capital letters as appropriate.
%%% Setting the entire title in upper-case letters is not correct, nor is
%%% capitalizing only the first letter of the title.

\title{Efficient Subspace Rendering for Dynamic Deformable Hands}

%%% Define the author list in the ``\author'' command. The ``\thanks''
%%% field can be used to define an e-mail address for the author.
%%% The ``\pdfauthor'' field should contain a comma-separated list of the
%%% authors of the paper, and is used, along with the title and keyword
%%% data, for PDF metadata. (To see this metadata, open the PDF in Adobe
%%% Reader and select ``File > Properties > Description.''

\author{TODO\\$^{1}$ McGill University \hspace{20pt} $^{2}$ Universit\'e de Montr\'eal \hspace{20pt} $^{3}$ Universidade de Coimbra}
\pdfauthor{TODO}

%%% User-defined keywords.

\keywords{radiosity, global illumination, constant time}

%%% End of the document preamble, start of the document.

\begin{document}

%%% A ``teaser'' image appears below the title and affiliation and above
%%% the two-column body of the paper. This is optional, but if you wish
%%% to include such an image, the commented-out code, below, can be used
%%% as an example. Please note that the inclusion of a ``teaser'' image
%%% may move the copyright space to the bottom of the right-hand column
%%% on the first page of your formatted output. This is acceptable.

\teaser{
   \includegraphics[height=1.5in]{images/teaser.pdf}
   \caption{\label{fig:teaser}TODO}
 }

%%% The ``\maketitle'' command uses the author and title information
%%% defined above, and prepares the formatted title.

\maketitle

%%% The ``abstract'' environment should contain the abstract for your
%%% content -- one to several paragraphs which describe the work.

\begin{abstract}
This is where the abstract goes.
\end{abstract}

%%% The ``CRCatlist'' environment defines one or more ACM ``Computing Review''
%%% (or ``CR'') categories, used for indexing your work. For more information
%%% on CR categories, please see http://www.acm.org/class/1998.

\begin{CRcatlist}
  \CRcat{I.3.3}{Computer Graphics}{Three-Dimensional Graphics and Realism}{Display Algorithms}
  \CRcat{I.3.7}{Computer Graphics}{Three-Dimensional Graphics and Realism}{Radiosity};
\end{CRcatlist}

%%% The ``\keywordlist'' prints out the user-defined keywords.

\keywordlist

%%% If you are preparing a paper to be presented in the Technical Papers
%%% program at one of our annual flagship events (and, therefore, using
%%% the ``annual'' parameter to the ``\documentclass'' command), the
%%% ``\TOGlinkslist'' command prints out the list of hyperlinked icons.
%%% If you are using any other parameter to the ``\documentclass'' command
%%% this command does absolutely nothing.

\TOGlinkslist

%%% The ``\copyrightspace'' command will leave clear an amount of space
%%% at the bottom of the left-hand column on the first page of your paper,
%%% according to the parameter used in the ``\documentclass'' command.

\copyrightspace

%%% The first section of your paper.

\section{Introduction}
\label{sec:introduction}

Hands are important. Hands are in the uncanny valley. Hand deformations are challenging because they combine large- and small-scale deformations and wrinkles. Rendering realistic deformable hands using real-world lighting is a difficult problem: local sub-surface scattering effects and global shadowing effects combine to make this hard.

Luckily, the space of hand deformations is much more constrained than that of arbitrary deformation and animation. We will exploit this to accelerate rendering.

We will analyse the subspace of plausible hand deformations and tailor a compact and realistic appearance model for hands that operates entirely within this subspace.

\section{Previous Work}
\label{sec:previous}

Most character animation in interactive applications is based on Skeletal-Subspace Deformation methods (see \cite{Magnenat-thalmann}). Several methods have been proposed providing finer results for hand meshes. Physically based muscle models have been used to model movement of skin and bones \cite{Fernando:2004:GGP:983868}. Data-driven methods have also been applied successfully, such as using eigenvectors to enhance SSD capabilities at very little cost \cite{Kry:2002:ERT:545261.545286}, or a set of control points to decouple the overall shape and wrinkle reconstructions \cite{Huang:2012:DCD:2311637.2311785}. Yet, how to realistically light these hand meshes is left untouched.

Skin is constituted by layers, and none is perfectly opaque. As such, subsurface scattering should be taken into account. Tattoos, freckles and veins are specially affected by this effect. While there are hardware based techniques for real time rendering of subsurface scattering \cite{Fernando:2004:GGP:983868}, they rely on approximations which are mostly intended for coarse effects, making them ineffective for these fine details.

Variations in the blood flow are also responsible for strong variations in skin color. Significant work has occurred for the rendering of this variation. Simulation of melanin and hemoglobin levels has been successfully used to represent changes of color due to facial expressions, emotions \cite{Jimenez:2010:PAM:1882261.1866167} or pressure \cite{Donner:2008:LHR:1409060.1409093}. Yet, this requires the capture of melanin and hemoglobin distribution from \emph{in vivo} scenarios, and has limited usability on interactive environments.


Additional note: \cite{Heidrich:2000:IMG:344779.344984} as information on how to reconstruct geometry from a normal map and precompute it's visibility. Still not sure what to make of it.

\section{Deformable Subspace Appearance Model}
\label{sec:approach}

This is where we describe our technique. We can break up the exposition according to our analysis of the subspace (Section~\ref{sec:subspace}), and our rendering model (Section~\ref{sec:model}).

\subsection{Deformable Hand Subspace}
\label{sec:subspace}

Formulate and illustrate the low-dimensionality of hand deformation. Then, do the same for the hand shading space.

\subsection{Subspace Shading}
\label{sec:model}

Reflection equation and PRT 101. LDPRT. SSSSS. Our hybrid model, including the env map segmentation. Our RBF model (and maybe discuss why linear wasn't good enough; compare the two).

\section{Implementation and Results}
\label{sec:results}

Discuss our implementation. Show our results.

\section{Conclusion}
\label{sec:conclusion}

This is where we repeat the introduction, except in past tense :)

\begin{comment}
\section*{Acknowledgements}
NSERC. FQRNT. Anybody else?
\end{comment}

%%% Please use the ``acmsiggraph'' BibTeX style to properly format your
%%% bibliography.

\bibliographystyle{acmsiggraph}
\bibliography{paper}
\end{document}
